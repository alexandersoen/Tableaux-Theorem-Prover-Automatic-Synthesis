\documentclass[onecolumn]{report}

\usepackage[margin=1in]{geometry}
\usepackage{parskip}
\usepackage{cite}
\let\oldbibliography\bibliography% Store \bibliography in \oldbibliography
\renewcommand{\bibliography}[1]{{%
  \let\chapter\section% Copy \section over \chapter
  \oldbibliography{#1}}}% Old \bibliography

\title{COMP2560 - Project Summary}
\author{Alexander Soen}

\begin{document}

\maketitle

\section*{Project}

\subsection*{Title}

Automated Synthesis of a Tableaux Theorem Prover for Classical Propositional
Logic using Coq.

\subsection*{Supervisor}

Rajeev Gore.

\section*{Objective}

It is essential to prove or disprove that a formula is a theorem in many areas
of applied logic. This is done through automatic reasoning, theorem provers.
One such versatile and successful type of theorem provers are tableau-based
theorem provers. These are theorem prover that utilise the semantic tableau, a
procedure which can determine if a formula is satisfiable, and hence if the
formula is a theorem. We aim to build a tool that given a set of tableau rules,
among other parameters, it will automatically generate a corresponding
tableau-based theorem provers which is proven to be correct. We aim to do this
for classical propositional logic first in the hopes that this can be extend
for more complex logical systems in the future.

\section*{Current Limitations}

Currently, the Tableau Work Bench \cite{twb}, implemented in O'Caml, exists as
a ``user-friendly framework for building automated tableau-based theorem
provers". It allows users to give an input set of rules for their custom
logical system in propositional logic to generate a corresponding tableau-based
theorem prover. However, the generated tableau-based theorem provers generated
are not guaranteed correctness.

\par

Apart from the Tableau Work Bench, to implement
a theorem prover for a custom logic system, assuming there is no pre-existing
one already, one would have to implement and program the theorem prover from
scratch.

\par

Furthermore, if one wants to prove that a logical set of rules corresponds to
a particular logical system they would have to prove this separately from their
corresponding implementation of a theorem prover. This is a major issue on
knowing if the implementation of the theorem prover correctly corresponds to
the logical system the user wants it to describe.

\section*{Approach}

Instead of implementing a framework for building automated tableau-based
theorem provers in O'Caml like the Tableau Work Bench, we instead implement a
similar system in Coq, a theorem prover itself. By using Coq to implement the
tool, we are able to prove properties about the process of proving or
disproving if a formula is a theorem. 

\par

This allows users to complete a completeness and soundness proof of their
logical rules (with respect to what they want these rules to describe) inside
Coq (proving the rules in-fact describe what they believe they do). This
approach gives us a tool which is capable of accepting an input to describe a
custom logical calculi, generating a corresponding tableau-based theorem
prover, and allows the user to be sure that the generated tableau-based theorem
prover does in fact correspond to their described logical calculi through
proof.

\section*{Motivation}

The reason we want a tool to build automated tableau-based theorem provers is
so that researchers who are interested in experiment with automated theorem
provers for calculi they are designing do not have to program their own
theorem prover. This allows users with limited programming experience to create
theorem provers for their calcui. By formally verifying the process of
generating the tableau-based theorem provers, the users of the tool are
guaranteed that the produced theorem prover expresses the calculi they
described.

\par

Currently, we require the user to prove that their logical rules are sound and
complete within Coq to ensure that the generated theorem prover expresses the
calculi the user describes (otherwise it generates a theorem prover with
no guarantee). However, as we have implemented the generation of the theorem
prover in Coq, the user can prove the properties of soundness and completeness
of theorem in Coq. Thus, we couple the proof of correctness of the logical
system with the creation of the corresponding theorem prover

\par

Coupling proof and implementation of a theorem prover is a step up from current
practices of trusting implementations, where the user need to check that the
theorem prover is correct (automatically done by us once the soundness and
completeness is proven). This may seem like additional work, that is to prove
soundness and completeness, but for those who want to test custom logical
systems, these proofs are necessary for their logical system to be useful.

\section*{Milestones}

\begin{table}[h]
\centering
\begin{tabular}{|p{0.65\textwidth}|p{0.35\textwidth}|}
\hline
Milestone & Date of Completion\textbackslash to Complete \\
\hline \hline
Background Learning: How to use Coq& 10 July, Semester Break \\
\hline
Background Learning: How a Tableau Proof Works& 17 July, Semester Break \\
\hline
Background Learning: How Sequent Calculus Works& 24 July, Week 1 \\
\hline
Encode Sequent Calculus and Tableau Calculus in Coq& 31 July, Week 2 \\
\hline
Prove Equivalence of Sequent Calculus and Tableau Calculus in Coq& 07 August, Week 3 \\
\hline
Prove Admissibility of Structure Rules in Sequent Calculus& 14 August, Week 4 \\
\hline
Implement Partition Application of Tableau Rules in Coq& 21 August, Week 5 \\
\hline
Implement Rule Application of a Tableau Node in Coq& 28 August, Week 6 \\
\hline
COMP2560: Project Summary& 01 September, Week 6 \\
\hline
Implement Tactic Language for Rule Application in Coq& 04 September, Mid-Semester Break \\
\hline
Implement Tree Search and Prove Correctness in Coq& 11 September, Mid-Semester Break \\
\hline
COMP2560: Paper Draft& 10 October, Week 10 \\
\hline
COMP2560: Paper Reviews& 17 October, Week 11 \\
\hline
COMP2560: Poster, Rebuttal& 24 October, Week 12 \\
\hline
COMP2560: Presentation& 27 October, Week 12 \\
\hline
COMP2560: Final Paper& 9 November, Exam Period\\
\hline
\end{tabular}
\end{table}

\bibliographystyle{unsrt}
\bibliography{reference}

\end{document}
