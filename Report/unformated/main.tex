documentclass{article}

\usepackage{amsmath}
\usepackage{amsthm}
\usepackage{amsfonts}
\usepackage{amssymb}
\usepackage{mathtools}
\usepackage{listings}
\usepackage[margin=1in]{geometry}
%\usepackage{parskip}
\usepackage{multicol}
\usepackage[english]{babel}
\usepackage[utf8]{inputenc}

\setlength{\parindent}{0pt}
\setlength{\parskip}{\baselineskip}

\newtheorem{definition}{Definition}
\newcommand{\eref}[1]{Equation \ref{#1}}

\lstset{
inputencoding=latin1
}

\title{Automated Synthesis of a Tableaux Theorem Prover for Classical
Propositional Logic using Coq}
\author{Alexander Soen}

\begin{document}

\begin{abstract}
I prove the equivalence of tableau calculus and sequent calculus.
\end{abstract}

\section{Introduction}

It is essential to prove or disprove that a formula is a theorem in many areas
of applied logic. This is done through automatic reasoning, theorem provers.
One such versatile and successful type of theorem provers are tableau-based
theorem provers. These are theorem prover that utilise the semantic tableau, a
procedure which can determine if a formula is satisfiable, and hence if the
formula is a theorem. We aim to build a tool that given a set of tableau rules,
among other parameters, it will automatically generate a corresponding
tableau-based theorem provers which is proven to be correct. We aim to do this
for classical propositional logic first in the hopes that this can be extend
for more complex logical systems in the future.

In this project we have the following,

\begin{enumerate}
\item Explored the formalisation of the equivalence of sequent calculi and 
tableau calculi in Coq.
\item Created a tableau-based theorem prover for classical
propositional logic using Coq.
\end{enumerate}

In the first part of the project, the formalisation of sequent calculi and
tableau calculi was explored and how it could be encoded into Coq. Sequent
calculi, like tableau calculi, can be utilised to determine whether a formula
is satisfiable. Gentzen first introduced the ``calculus of sequents" in
``Untersuchungen uber das logische Schliessen" (Investigations into logical
deduction - gerhard gentzen)). It is now a common system to use when reasoning
about satisfiability of formula in classical propositional logic. It follows
that there is a strong relationship between the notion of a closed tableau in
the tableau calculus and a sequent being derivable in the calculus of sequents.
By exploring the equivalence of sequent calculus and tableau calculus, a
transformation from the semantics of sequent calculus and tableau calculus can
be made in Coq. With this, users of our tool, the automatic synthesiser of
tableau-based theorem provers, will not need prerequisite knowledge about the
tableau calculus to use the tool. Instead, they can use the semantics of the
common sequent calculus.
(Not too sure if this is correct. Need more citations)

In the second part of the project, we attempt to make a generalised synthesiser
for logical systems. We aim to make a tool kit in Coq which allows users to
input logical rules and then extract a corresponding theorem provers which is
proven to be correct with respect to the logical system the user proves the
logical rules corresponds to.

This project is aimed to extend the work done in the Tableau Work Bench
(ADD CITATION - Pietro). The Tableau Work Bench, implemented in O'Caml, exists
as a ``user-friendly framework for building automated tableau-based theorem
provers". It allows users to give an input set of rules for their custom
logical system in propositional logic to generate a corresponding tableau-based
theorem prover. However, the generated tableau-based theorem provers generated
are not guaranteed correctness.

Apart from the Tableau Work Bench, to implement a theorem prover for a custom
logic system, assuming there is no pre-existing one already, one would have to
implement and program the theorem prover from scratch.

Furthermore, if one wants to prove that a logical set of rules corresponds to
a particular logical system they would have to prove this separately from their
corresponding implementation of a theorem prover. This is a major issue on
knowing if the implementation of the theorem prover correctly corresponds to
the logical system the user wants it to describe.

We aim to developed a tool which will still allow users to quickly prototype
and implement calculi they wish to test like the Tableau Work Bench while also
providing an option to have the implementation proven to be correct.
To simplify, in this project we only look at classical propositional logic.

\section{Background}

\subsection{Tableau Calculus}

A tableau calculus provides a decision procedure to determine if a formula
is satisfiable through the decomposition of sets of formulae. Further more,
the tableau method can be used to determine if a formula is valid in a specific
logical system.

More specifically, a tableau calculus consists of a finite set of rules which
describe the logical system, $L$. The tableau method can determine if formulae
are valid with respect to $L$. Underlying the tableau method is a tree
structure, where branches represent a rule application.

\begin{definition}{Tableau Rule}

The rules of a tableau are expressed as sets, multi-sets or lists depending on
the logics being expressed. We will express the tableau rules as a set as we
are working within classical propositional logic. A rule is composed of a
numerator and a denominator. A numerator $\mathcal{N}$ is a set of formulae in
the logical system $L$. A denominator is either a set of branches,
$\mathcal{D}_i$, which are each sets of formulae in $L$ or the symbol $\bot$
signifying a closed tableau, indicated the termination of a branch. These rules
are typically written as the following,

\begin{multicols}{2}
\begin{equation*}
(\rho)\frac{\mathcal{N}}{\mathcal{D}_1 \vert \cdots \vert \mathcal{D}_n}
\end{equation*}
\break
\begin{equation*}
(\rho^{\prime})\frac{\mathcal{N}}{\bot}
\end{equation*}
\end{multicols}

Each tableau rule has a set of main formulae which dictate the way the rule
gets applied. These formulae are denoted as the \textit{principal formulae} of
the rule. The rest of the formulae are denoted as the \textit{side formulae} of
the rule.

To apply a rule to a formula set $\Gamma$, the variables in the numerator
$\mathcal{N}$ must be unified to match $\Gamma$.  Then the denominator must be
instantiated following the unification of the numerator. Each branch of the
denominator acts as subgoals.

\end{definition}

A tableau for a formula set $\Gamma$ is a tree of nodes where $\Gamma$ is the
root and all children of a node are applications of a rule on that node.

\begin{definition}{Invertible Rule}
A rule $\rho$ is invertible if and only if whenever there exists a closed
tableau for an instance of its numerator, there exists a closed tableau for
each branch in its denominator.
\end{definition}

A formula can be shown to be unsatisfiable if a tableau exists in which all
branches have been expanded to be closed. Thus to show if a formula is valid in
classical propositional logic, we show that its negation is unsatisfiable in
tableau.

\subsection{Sequent Calculus}

A sequent calculus is a very useful tool in proof theory. Proving the validity
of formulas in a sequent calculus is similar to the tableau method, however the
node in a sequent calculus proof are a sequent of formulae instead of a set of
formulae.

\begin{definition}{Sequent}
A sequent is an expression of the following form,

\begin{equation*}
\Gamma ==> \Delta
\end{equation*}

Where $\Gamma$ and $\Delta$ are a set, multi-set or list of formulae depending
on the calculi being defined. $==>$ acts as an auxiliary symbol. We will
express a sequent with lists.

$\Gamma$ is denoted as the antecedent and $\Delta$ is denoted as the succedent.
Both of these expressions could be empty.

Given the sequent,

\begin{equation*}
A_1, \ldots, A_n ==> B_1, \ldots, B_n
\end{equation*}

This is equivalent to the following formula,

\begin{equation*}
(A_1 \& \ldots \& A_n) \supset (B_1 \vee \ldots \vee B_n)
\end{equation*}

\end{definition}

(Define more, seq rules, deduction tree, cut?)

\subsection{Classical Propositional Logic}

(Define tableau + sequent defs)

\section{Proof Theory}

The system of sequent calculus has been encoded into Coq based on the rules
given by Floris van Doorn. First the notion of a sequent being a tuple of lists
is defined, the left side and right side of a sequent. Given this, we encode
the notion of sequent being derivable as a direct translation of the sequent
rules.

Furthermore, the system of tableau calculus was encoded in a similar manner.
A tableau is represented as a list. Then the notion of a closed tableau is
established through a direct translation of the tableau rules.

To show that the system of tableau calculus is equivalent to the system of
sequent calculus we aim to prove the following,

\begin{equation}
\text{closed }X \iff X=\Gamma \cup \neg \Delta \, \& \, \text{derivable }
\Gamma ==> \Delta
\label{tableau sequent equivalence}
\end{equation}

This is proven to an extent in Coq. Currently the proof is reliant on the
exchange lemma for the tableau calculus begin admitted.

The proof is also reliant on a slight modification of rules. However,
equivalence of the rules with respect to the ones being used by Floris van
Doorn can be achieved through the admissibility of the weakening lemma.

The admissibility of the structure rules, apart from the cut lemma, has been
proven with Floris van Doorn's rules for sequent calculus by relying on a
proof of the exchange lemma being admitted.

\section{Tableau-based Theorem Prover}

We define the process of creating a tableau tree through the recursive
application of rules on nodes. This function for this process is then extracted
using Coq's inbuilt tools to give a program which can be used to check if a
formulae is valid with respect to the rules defined in Coq.

\subsection{Data Structures}

To implement the tableau-based theorem prover for classical propositional
logic, we define the following data structures. We make a concious decision to
define the data structures we use inside $Type$ in Coq. This is because in the
extraction of programs provided by Coq, anything defined in $Prop$ (and not
$Type$) get discarded and subsequently not extracted to a program.

We define a propositional logic formulae similarly to the definition found in
Doorn's work (Reference Propositional Calculus in Coq).

\begin{lstlisting}
Inductive PropF : Type :=
 | Var : string -> PropF
 | Bot : PropF
 | Neg : PropF -> PropF
 | Conj : PropF -> PropF -> PropF
 | Disj : PropF -> PropF -> PropF
 | Impl : PropF -> PropF -> PropF
.
\end{lstlisting}

We use $string$ to represent propositional symbols in our formulae. With this,
$PropF$ can be proven to be decidable under equality using the already proven
lemma that strings are decidable, $string\_dec$.

General data structures for the tableau calculus are defined using $PropF$.
The general data structures are mainly for defining a node of a tableau and
the rules of a tableau system.

\begin{lstlisting}
Inductive Results :=
  | Closed
.

Definition PropFSet := list PropF.
Definition Numerator := PropFSet.
Definition Denominator := sum (list PropFSet) Results.
Definition Rule := prod Numerator Denominator.
Definition TableauNode := sum PropFSet Results.
\end{lstlisting}

Notably, we define a $Rule$ as only the principal formulae of a numerator and
denominator of a tableau rule.

The decidability of each of these types are proven using induction.

We define the explicit derivation tree of a tableau proof. That is the tree
of tableau nodes where each child of a node generated from an application of a
tableau rule.

\begin{lstlisting}
Inductive DerTree :=
  | Clf : DerTree
  | Unf : PropFSet -> DerTree
  | Der : PropFSet -> Rule -> list DerTree -> DerTree
.
\end{lstlisting}

The data structure is similar to a general rose-tree. The main difference is
that there are two possible leaf choices.  One type of a leaf represents a
closed branch in the tableau tree. The other type of a leaf represents a node
which is not closed and possibly could be expanded by additional rule
applications. The inner nodes of a $DerTree$ also contains information about
the rule which is applied to generate its children. We define that a tableau
is closed if there exists a $DerTree$ which is generated through correct rule
applications and all of its leaves are closed, of type $Clf$.

Furthermore, as $DerTree$ was defined recursively as a list of $DerTree$s, Coq
was not able to infer a particularly useful induction scheme, that is an
inductive principle used to prove properties about the type $DerTree$.

We prove that the following induction scheme is valid for $DerTree$.

\begin{lstlisting}
Fixpoint DerTree_recursion (PT : DerTree -> Type) (PL : list DerTree -> Type)
  (f_Clf : PT Clf) (f_Unf : forall x, PT (Unf x))
  (f_Der : forall x r l, PL l -> PT (Der x r l))
  (g_nil : PL nil)
  (g_cons : forall x, PT x -> forall xs, PL xs -> PL (cons x xs)) 
  (t : DerTree) : PT t.
\end{lstlisting}

Simply this recursion scheme is just how one would typically do structural
induction on a rose tree, however the proposition for the branches of an inner
node must be explicitly stated. That is, it formally states that for some
property $P$, if the property is true for any type of leaf node and if $P$
holds for all the children of an inner node implies it holds for the inner
node, then the property $P$ is true for all $DerTree$s.

Again the decidability of $DerTree$ is proven using the previous
decidability properties and induction using $DerTree\_recursion$.

\subsection{Implementing a Theorem Prover in Coq}

The main difference when implementing a theorem prover in Coq than implementing
a theorem prover in another programming language like O'Caml is that all the
functions must be proven to be terminating. Functions defined in Coq can be
classified in two main groups: functions defined with primitive recursion and
functions defined with general recursion.

The case in which a function is defined with primitive recursion, Coq
automatically infers its termination requiring no additional input from the
user.

However, to define functions with general recursion the user must provide a
proof to justify its termination. This involves proving that a function has a
decreasing property in each of its recursive calls with respect to a
well-founded relation.

We use two main well-founded relations to prove termination for functions when
defining the theorem prover in Coq.

The first is that the relation of a list with respect to length is
well-founded.

\begin{lstlisting}
Definition lengthOrder (A : Type) (xs ys : list A) := length xs < length ys.

Lemma lengthOrder_wf : forall A, well_founded (lengthOrder A).
\end{lstlisting}

The second is that relation of $DerTree$ with respect to depth is well-founded.
The depth of a $DerTree$ is defined as the following.

\begin{lstlisting}
Fixpoint depthDerTree (T : DerTree) :=
  match T with
  | Unf _ => 0
  | Der _ _ branch => 1 + maxList (map depthDerTree branch)
  | Clf => 0
  end.
\end{lstlisting}

Where $maxList$ returns the maximum value of a $list\;nat$. Hence, we have the
well-founded lemma for $depthOrder$.

\begin{lstlisting}
Definition depthOrder T1 T2 := depthDerTree T1 < depthDerTree T2.

Lemma depthOrder_wf : well_founded depthOrder.
\end{lstlisting}

The two relations were proven to be well-founded through regular induction on
lists as provided by Coq for $lengthOrder$ and induction using
$DerTree\_recursion$ for $depthOrder$. For the $depthOrder$ proof, the explicit
branch proposition used was:

$$P(bs) = \forall b,\; depthDerTree\; b < S\;(maxList\; (map\; depthDerTree
\;bs)) \rightarrow Acc\; depthOrder\; b$$

Where $Acc$ is as defined in Coq.

In the subsequent definitions of function defined using general recursion, the 
well-founded relation used to define the function will be stated and the
function will be defined as if it was a $Fixpoint$ function in this paper with
keyword $GFixpoint$ instead of $Fixpoint$, omitting its actual definition in
Coq with the $Fix$ operator for clarity.

\subsection{Generating a Tableau Through Rule Application}

To generate a $DerTree$ which expresses a correct tableau, the application of
a rule must be defined. The overall process we want defined is that given an
initial leaf node with formulae set $\Gamma$, $Unf\; \Gamma$, through a series
of rule applications a new $DerTree$ is given which defines a tableau which can
be determined to be closed or not. Thus determining if $\Gamma$ is a
satisfiable set of formulae or not assuming the application of rules is
exhaustive.

We first define how a rule is applied to a $PropFSet$ type before moving on
to working with a $DerTree$. When applying a rule to node, a $PropFSet$ which
is not closed, we must consider the following,

\begin{enumerate}
\item If the rule is applicable to the node.
\item Which formulae in the node is the rule being applied to.
\end{enumerate}

To check if a rule is applicable to a node, we must check if there exists a map
from propositional variables to formulae such that the rule's numerator under
the map is equal to a set of formulae in the node.

\begin{definition}
An instance of a rule is denoted as the evaluation of a map to its numerator
and denominator such that each of its propositional variables are instantiated.
Generally, a formula $p$ is and instance of a formula $q$ if there exists a
map $\phi$ from propositional variables to formulae such that $\phi(p) = q$.

We denote that a rule $R$ can be applied to a set of formulae $\Gamma$ if there
exists an instance $R$ such that each of the formulae in the numerator of the
instance of $R$ exists in $\Gamma$. An instance of this type is denoted as an
applicable instance of $R$ for set $\Gamma$.

A map to give an instance of a rule is defined as a partition. The set of
applicable instances of $R$ for a set $\Gamma$ is denoted as the set of
applicable partitions.
\end{definition}

If we consider all maps which allow a rule to be applied to a node, this is the
combination of all the possible formulae choice we have when applying a rule.
We want to get all possible applications of a rule in node to generate all the
possible ways a tableau can be expanded.

An application of a rule $R$ to a set of formulae $\Gamma$ is defined as when
all formulae in the numerator of an applicable instance $I$ of $R$ for the set
$\Gamma$ is replaced by the denominator of $I$ for each branch.

To define how a rule is applied to a set of formulae $PropFSet$ in Coq, we
first define how to generate the list off all applicable partitions for a rule
with respect to the $PropFSet$. The notion of a partition is defined as the
following.

\begin{lstlisting}
Definition Partition := list (PropF * PropF).
\end{lstlisting}

The method of finding the applicable partitions with respect to a numerator of
a rule $R$ and a set of formulae $\Gamma$ is done in an iterative method. For
each formulae in the numerator of a rule, we check if there is an instance of
it in $\Gamma$. We extend this partition for each of the formulae in the
numerator of $R$. This process give us a list of partitions which are all the
applicable partitions for the rule $R$ and set $\Gamma$. When extending
partitions to account for additional formulae in the numerator, we also need to
make sure that the partitioning map is consistent. That is if we already map a
propositional variable $a$ to a formula $p$, none of the propositional
variables in $p$ gets mapped as a result of the rest of the partition.  We
implement a function in Coq that generates all the possible consistent
partitions with respect to two $PropFSet$s as the following.

\begin{lstlisting}
GFixpoint getPartitions_help
  (schema propset : PropFSet) (acc : Partition) : list Partition :=
  match schema with
  | nil => acc :: nil
  | s :: ss => let Pi := partition_help s propset acc in
  flat_map (fun pi => getPartitions_help (applyPartition ss pi) propset pi) Pi
  end.

Definition getPartitions schema propset :=
  (getPartitions_help schema propset nil).
\end{lstlisting}

Where $partition\_help$ finds all the maps of instances of $s$ in $propset$ 
which can extend the partition $acc$ consistently in each of the recursive 
calls of $getPartitions\_help$. $applyParitition$ simply applies the partition
$pi$ to all formulae in $ss$, updating the propositional variables in propset
every recursive call.

$getParition\_help$ had to be defined with the $lengthOrder$ relation to prove
that the $applyPartition \; ss \; pi$ decreases in size every recursive call to
prove termination.

Given a function to generate partitions, we now wish to define a function which
allow us to apply rules to the leaves of a $DerTree$ data structure, that is
expand out a tableau tree. We define the nodes in the $DerTree$ which can have
rules applied to as the leaves with the $Unf$ constructor.

We first define a function which will apply a function from $DerTree$ to
$DerTree$ to the $n^{\text{th}}$ un-closed leaf of a $DerTree$ data structure. 
That is we define a function which will apply a change to and un-closed leaf
but leave the rest of the tree the same. We denote the $n^{\text{th}}$
un-closed leaf as the $n^{\text{th}}$ goal which needs to be satisfied when
attempting to show a formulae set is unsatisfiable. 

To operate on the $n^{\text{th}}$ goal of a $DerTree$ we need to determine
how to traverse a $DerTree$ to get to the $n^{\text{th}}$ goal. To do this, we
define a function to get the goals of a $DerTree$ as the following.

\begin{lstlisting}
Fixpoint getGoals (T : DerTree) : list DerTree :=
  match T with
  | Der _ _ branches => match branches with
                        | nil => nil
                        | _ => flat_map getGoals branches
                        end
  | Clf => nil
  | Unf _ => T :: nil
  end.
\end{lstlisting}

The function ignores all closed leaves as goals as the goal to show that a
formula is unsatisfiable is to make all the leaves of the expansion of the
tableau closed, that is the closed leaves has already satisfied this
condition. To traverse the tree, we must determine which branch to go down when
starting from an inner node. To do this, the number of goals in each branch are
checked.

\begin{lstlisting}
Fixpoint traverseToNG_help
  (Ts : list DerTree) (n : nat) (acc : list DerTree) :=
    if gt_dec n (length Ts) then None else (
    if eq_nat_dec n 0 then None else (
    match Ts with
    | nil => None
    | x::xs => match x with
               | Clf => traverseToNG xs n (acc ++ (x::nil))
               | _ => let xchild := (length (getGoals x)) in
                      if le_dec n xchild then Some (acc, x, xs, n)
                      else traverseToNG xs (minus n xchild) (acc ++ (x::nil))
               end
    end)).

Definition traverseToNG (Ts : list DerTree) (n : nat) :=
  traverseToNG_help Ts n nil.
\end{lstlisting}

$traverseToNG$ defines a function which given a list of branches from an inner
node and a goal number will return information about the branches with respect
to the aimed goal. Assuming that the list of branches has an $n^{\text{th}}$
goal, $traverseToNG$ returns a tuple containing the branches before the desired
goal, the branch containing the $n^{\text{th}}$ goal, the branches after the
desired goal and the updated goal number with respect to the branch which
contains the goal.

(Might add proof to show that this is indeed true)

With $traverseToNG$, we define a function to apply a function from $DerTree$ to
$DerTree$ on the $n^{\text{th}}$ goal.

\begin{lstlisting}
GFixpoint toBranchNG (T : DerTree) (f : DerTree -> DerTree) (n : nat) :
  DerTree :=
  match T with
  | Der propset rule branches => match traverseToNG branches n with
                                 | None => T
                                 | Some (acc, x, xs, n') =>
                            Der propset rule (acc ++ (toBranchNG x f n') :: xs)
                                 end
  | _ => f T
  end.
\end{lstlisting}

The function simply recursively traverses a $DerTree$ until a leaf is reached
in which its argument $f$ will be evaluated to the leaf giving a new tree.
We also define a function to get the $n^{\text{th}}$ goal in a $DerTree$.

\begin{lstlisting}
GFixpoint getNGoal (T : DerTree) (n : nat) :=
  match T with
  | Der _ _ branches => match traverseToNG branches n with
                        | None => None
                        | Some (_, x, _, n') => getNGoal x n'
                        end
  | _ => if eq_nat_dec n 0 then None else Some T
  end.
\end{lstlisting}

These two function are defined using the $depthOrder$ relation to show that the
$DerTree$ argument being passed down every recursive call is decreasing.

With these two functions, the following function is defined to apply a rule to
the $n^{\text{th}}$ goal.

\begin{lstlisting}
Definition updateLeaf (T : DerTree) : (DerTree -> DerTree) := fun _ => T.

Definition applyRtoNG (T : DerTree) (rule : Rule) (n : nat) : list DerTree :=
  match getNGoal T n with
  | None => None
  | Some goal => 
    match goal with
    | Unf propset =>
      match getPartitions (getNumerator rule) propset with
      | nil => None
      | Pi => match optionSucMap _ _ (applyPartitionRuleD rule propset) Pi with
              | None => None
              | Some newNodes =>
              Some (map (fun x => toBranchNG T x n) (map updateLeaf newNodes))
              end
      end
    | _ => None
    end
  end.
\end{lstlisting}

Where $optionSucMap$ is a mapping functions which only returns successful
results, and if there are no successful results it returns $None$.
The function does the following,

\begin{enumerate}
\item Gets the $n^{\text{th}}$ goal of the $DerTree$ $T$, $G$.
\item Makes sure the goal is infact an un-closed leaf.
\item Generates all applicable partitions of the $Rule$ $rule$ with respect to
      $G$.
\item Generates all the expansions of the tree with respect to each applicable
partition.
\end{enumerate}

With $applyRtoNG$, we can now define a simple interactive theorem prover. When
we want to prove that a formula is unsatisfiable using tableau we need to
consider three variables when expanding a tableau $DerTree$.

\begin{itemize}
\item \textbf{Node-choice:} determining which of the goals in the $DerTree$ to
expand.
\item \textbf{Rule-choice:} determining which rule to apply to the goal.
\item \textbf{Formula-choice:} determining which formula to apply the rule to.
\end{itemize}

$applyRtoNG$ requires the user to specify the rule and the goal the tableau
expands to. It returns a list of $DerTree$s if the rule is applicable to the
goal. Each of these results are just the combination of formula-choices for
the application of a rule. Hence, by simply defining a function which allows a
user to selects one of these results, the user can drive a tableau proof by
take care of the possible choices when expanding a tableau. We define the
following function so that the user can select a formula-choice after
$applyRtoNG$ returns all possible formula-choices.

\begin{lstlisting}
Fixpoint pickNFApply (results : option (list DerTree)) (n : nat) : DerTree :=
  match results with
  | None => None
  | Some nil => None
  | Some (x::xs) => match n with
                    | 0 => None
                    | 1 => Some x
                    | S n' => pickNFApply (Some xs) n'
                    end
  end.
\end{lstlisting}

\subsubsection{Example of a User Guided Proof}

We demonstrate how $applyRtoNG$ and $pickNFApply$ can be used to prove the
validity of formulae in classical propositional logic by proving that the
following formula is valid.

\begin{equation}
(A \wedge (A \implies B)) \implies B
\label{cpl example formula}
\end{equation}

The following set of tableau rules for classical propositional logic are used
to prove this.

\begin{multicols}{3}
\begin{equation*}
(Id)\frac{p\,;\,\neg p\,;\,X}{\bot}
\end{equation*}
\break
\begin{equation*}
(\wedge)\frac{\phi \wedge \psi\,;\,X}{\phi\,;\,\psi\,;\,X}
\end{equation*}
\break
\begin{equation*}
(\vee)\frac{\phi \vee \psi\,;\,X}{\phi\,;\,X\,|\,\psi\,;\,X}
\end{equation*}
\end{multicols}

First, these rules are converted to the $Rule$ type in Coq, only including
the principal formulae of each tableau rule. We alias each of the rules as the
following in Coq.

\begin{lstlisting}
Inductive CRule := 
  | IdC : CRule
  | OrC : CRule
  | AndC : CRule.
\end{lstlisting}

With function $getCRule : CRule \rightarrow Rule$ which returns the $Rule$
tuple associated to each of the tableau rules aliased by the inductive type
$CRule$. To determine the validity of \eref{cpl example formula}, we look at
the formula's negation. The negation of \eref{cpl example formula} is converted
into negation normal form such that the rules defined are applicable, and thus
operate on a logically equivalent formula to the negation of \eref{cpl
example formula} to determine the satisfiability of the formula. The converted
formula to negation normal form is the following.

\begin{equation}
A \wedge ((\neg A \vee B) \wedge \neg B)
\label{cpl example nnf}
\end{equation}

Thus, with \eref{cpl example nnf}, we have the following proof in Coq for
tableau.

$$
Definition\; cpl\_example\; :=\; (\#"A"\, \vee\, ((\neg\, (\#"A"))\, \vee\, (\# "B"))\, \wedge\, (\neg\, (\# "B"))).
$$

\begin{lstlisting}
Definition step1 := Unf (cpl_example :: nil).
Definition step2 := pickNFApply_nil (applyCRtoNG step1 AndC 1) 1.
Definition step3 := pickNFApply_nil (applyCRtoNG step2 AndC 1) 1.
Definition step4 := pickNFApply_nil (applyCRtoNG step3 OrC 1) 1.
Definition step5 := pickNFApply_nil (applyCRtoNG step4 IdC 1) 1.
Definition step6 := pickNFApply_nil (applyCRtoNG step5 IdC 1) 1.
\end{lstlisting}

This generated the following $DerTree$ structure.

\newpage

(Different types of rule application, to SetPropF, DerTree etc)

(Well-founded Relations)

(Strategy language)

(Defining a terminating function for tree search)

(Backtracking and further properties)

\end{document}
