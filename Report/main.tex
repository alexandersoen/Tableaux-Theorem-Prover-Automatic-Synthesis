\documentclass{report}

\usepackage{amsmath}
\usepackage{amsthm}
\usepackage{amsfonts}
\usepackage{amssymb}
\usepackage{mathtools}

\title{ Automated Synthesis of a Tableaux Theorem Prover for Classical
Propositional Logic using Coq}
\author{Alexander Soen}

\begin{document}

\begin{abstract}
I prove the equivalence of tableau calculus and sequent calculus.
\end{abstract}

\section{Introduction}

\section{Proof Theory}

The system of sequent calculus has been encoded into Coq based on the rules
given by Floris van Doorn. First the notion of a sequent being a tuple of lists
is defined, the left side and right side of a sequent. Given this, we encode
the notion of sequent being derivable as a direct translation of the sequent
rules.

\par

Furthermore, the system of tableau calculus was encoded in a similar manner.
A tableau is represented as a list. Then the notion of a closed tableau is
established through a direct translation of the tableau rules.

\par

To show that the system of tableau calculus is equivalent to the system of
sequent calculus we aim to prove the following,

\begin{equation}
\text{closed }X \iff X=\Gamma \cup \neg \Delta \, \wedge \, \text{derivable }
\Gamma ==> \Delta
\label{tableau sequent equivalence}
\end{equation}

This is proven to an extent in Coq. Currently the proof is relent on the
exchange lemma for the tableau calculus begin admitted.

\par



\end{document}
