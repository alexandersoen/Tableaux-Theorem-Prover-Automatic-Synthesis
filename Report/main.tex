\documentclass{llncs}

\usepackage[utf8]{inputenc}
\usepackage{amsmath}
\usepackage{cite}
\usepackage{array, longtable}

\DeclareUnicodeCharacter{2227}{$\wedge$}
\DeclareUnicodeCharacter{2228}{$\vee$}
\DeclareUnicodeCharacter{AC}{$\neg$}

\newcommand{\nnf}{{nnf}}
\newcommand{\comment}[1]{}
\allowdisplaybreaks

\begin{document}

\title{Automated Synthesis of a Tableaux Theorem Prover for Classical
Propositional Logic using Coq}

\author{Alexander Soen}
\institute{Australian National University}

\maketitle
%
\begin{abstract}
Proving or disproving that a formula is a theorem in a logic is an essential
process in many areas of applied logic. A variety of logics utilize the tableau
method to implement theorem provers \cite{d2013handbook}. We describe a
framework which allows users to synthesise tableau-based theorem provers for
various logics. Unlike the current systems in the literature, we implement this
framework in Coq \cite{barras1997coq}, an interactive theorem prover. Similarly
to the literature's current tools for general logics, the framework allows the
user to input a description of a logic, including non-classical logics, to
produce a corresponding theorem prover. However, as the framework is
implemented in an interactive theorem prover, it can be used to prove various
properties about the synthesised theorem prover in Coq. The properties which
can proven include: the correct correspondence between the rule set and the
synthesis theorem prover, and the completeness and soundness of the rule set
with respect to the logic being defined. We demonstrate this framework in Coq
by implementing the standard tableau calculus for classical propositional
logic.
\end{abstract}
%
\section{Introduction}
%
Many different implementations of the tableau method exists within the
literature for specific logics. There exists some very effective domain
specific theorem provers such as Fact++ \cite{tsarkov2006fact++}
and MSPASS \cite{hustadt2000mspass}. However, whenever a new tableau calculus is
devised, trying to develop a corresponding theorem prover can be difficult
without specific knowledge in programming. The current systems which try and
accommodate for generic logic systems include LoTREC \cite{del2001lotrec} and
the Tableau Work Bench \cite{abate2007tableau}. LoTREC accepts a semantic
description of a logic to define a graph-based tableau
\cite{castilho1997modal}. The Tableau Work Bench allows the user to define
tableau rules and specify a strategy to guide the proof search procedure which
determines if a formula is a theorem. We choose to follow the Tableau Work
Bench's input language as it directly corresponds to the tableau calculus the
user describes.

Although both LoTREC and the Tableau Work Bench allows a user to describe the
tableau rules to define a logic as a tableau system, there are no guarantees
that the tableau rules correctly associates to the logic the user tries to
describe. Currently, the user must prove this correspondence separately to
using LoTREC or the Tableau Work Bench. However, even if the user does this,
there are still no guarantees that LoTREC or the Tableau Work Bench correctly
translates the set of rules such that the tableau method defined correctly
represents the logic. Furthermore, there is no way to prove properties
regarding proof search when using LoTREC or the Tableau Work Bench.

We aim to implement a system similar to the Tableau Work Bench which
additionally allows us to prove correctness of the rules which the user inputs
with respect to the logic they are trying to define. Furthermore, we want the
system to allow the user to prove properties about the tableau search procedure
with respect to the defined logic. We choose to implement the framework in Coq
for two main reasons.  The first reason is the extraction mechanism of Coq.
This allows us to transform Coq proofs and, importantly, functions into
functional programs \cite{letouzey2008extraction}.  This allows us to prove
various properties about the search procedure to determine if a formula is a
theorem in a logic defined in Coq and its corresponding extracted program. The
second reason is to couple the implementation and associated proofs of the
theorem prover. For example, within the Coq environment, a user can prove that
the rule set they give to the framework describes a specific logic. The
framework then can guarantee that the generated theorem prover correctly
corresponds to the logic the user is trying to describe with the rule set.

Although providing additional guarantees about the generated theorem prover
requires the user to provide proofs to the system, many of these proofs are
standard when defining a useful tableau calculi for a logic.  For example, to
show correctness of a proof search, the rules first must be shown to be sound
and complete with respect to the logic it is describing.  However, when
defining a tableau calculi for a logic, a proof of soundness and completeness
is needed to show that the tableau calculi correctly corresponds to that logic.
Hence, before the implementation of a theorem prover for a tableau calculi,
this property is already proven. For these logics, the proof of soundness and
completeness just need to be translated to Coq in the framework.

In this paper, we provide a description of the implementation of a framework
implemented in Coq which synthesises theorem provers when given a grammar of
the formulas in a logic and a set of corresponding tableau rules. We
demonstrate the framework with the standard grammar and tableau rules for
classical propositional logic.
%
\section{Preliminaries}
%
In this section we outline the basic concepts on tableau, tableau for classical
propositional logic and notation for the rest of the paper.
%
\subsection{Tableau Calculi}
%
A tableau calculus provides a decision procedure to determine if a formula is
satisfiable through the decomposition of sets of formulae. Furthermore, the
tableau method can be used to determine if a formula is valid in a specific
logic.

More specifically, a tableau calculus consists of a finite set of rules which
describe a logic, $L$. The tableau method can determine if formulae
are $L$-valid through repeated application of the rule set to
determine if the tableau is closed. Underlying the tableau method is a tree
structure, where branches from nodes to nodes represent a rule application.
%
\begin{definition}{Tableau Rule}\label{Tableau Rule}

The rules of tableau calculus consist of a numerator $\mathcal{N}$ and
either finitely many denominators $\mathcal{D}_i$ separated by vertical bars
(\ref{rule non-closed}), or a $\times$ symbol signifying a terminating branch
(\ref{rule closed}).  The numerator an denominators are sets of formulae.
%
\comment{
The rules of a tableau calculus are expressed as sets, multi-sets or lists
depending on the logic being expressed. We will express the tableau rules as a
list as we are primarily working with classical propositional logic. A rule is
composed of a numerator and a denominator. A numerator $\mathcal{N}$ is a set
of formulae in the logical system $L$. A denominator is either a set of
branches, $\mathcal{D}_i$, which are each sets of formulae in $L$ or the symbol
$\times$ signifying a closed tableau, indicated the termination of a branch.
These rules are typically written as the following where (\ref{rule
non-closed}) represents a rule with a denominator as a set of formulae and
(\ref{rule closed}) represents a rule which results in a closed branch,
}
%
\begin{multicols}{2}
\noindent
\begin{equation}
\frac{\mathcal{N}}{\mathcal{D}_1 \vert \cdots \vert \mathcal{D}_n}
\label{rule non-closed}
\end{equation}
\begin{equation}
\frac{\mathcal{N}}{\times}
\label{rule closed}
\end{equation}
\end{multicols}
%
Each tableau rule has a set of main formulae which dictate the way the rule
gets applied. These formulae are denoted as the \underline{principal formulae}
of the rule. The rest of the formulae are denoted as the \underline{side
formulae} of the rule.

The numerator and denominators of the rules are often represented as sets,
multi-sets or lists depending on what logic is being represented by the tableau
calculus. In this paper we will primarily be using the list representation as
we are working with classical propositional logic.
\end{definition}
\begin{definition}{Tableau Calculus}

A tableau calculus is defined to be a finite set of tableau rules.
\end{definition}
\begin{definition}{Tableau}

A tableau for a formula set $\Gamma$ is a tree of nodes where $\Gamma$ is the
root and all children of a node are applications of a rule on that node.
\end{definition}
\begin{definition}{Closed and Open Tableau}

We describe the relation between a closed tableau and satisfiability with
respect to the tree in which a tableau expresses.

A tableau is closed if all the leaves of the tableau tree have the symbol
$\times$.  If a tableau is not closed, then it is open.
\end{definition}
\begin{definition}

A formula $\Gamma$ is unsatisfiable if there exists a tableau of $\Gamma$ which
is closed.
\end{definition}
To apply a rule to a formula set $\Gamma$, the variables in the numerator
$\mathcal{N}$ must be unified to match $\Gamma$. This can only occur if there
is a set formulae in $\Gamma$ which is an instance of the numerator of the
rule. Then the denominator must be instantiated following the unification of
the numerator. Each branch of the denominator act as sub-goals in showing the
satisfiability of $\Gamma$.

We formalise the unification process when applying a rule to a formula as the
following.
\begin{definition}{Partition}\label{Partition}

We define a partition to be a map from propositional variables, $PV$, to
formulae, $F$, in a logic. Given a partition $\pi$, we define the evaluation of
$\pi$ on a formula $f$, $\pi(f)$, to be the substitution of the propositional
variables in $f$ with respect to $\pi$. For example, in classical propositional
logic suppose we have a partition $\pi$, which maps the propositional variable
$p$ to the formula $q \wedge r$, and a formula $f = p \vee p$. Then $\pi(f) =
(q \wedge r) \vee (q \wedge r)$.
\end{definition}
\begin{definition}{Formulae Instance}\label{Formulae Instance}

A formula $p$ is an instance of another formulae $q$ if there exists a
partition $\pi$ such that $\pi(q) = p$.

A set of formulae $\Gamma$ is an instance of another set of formulae
$\Delta$ if there exists a partition $\pi$ such that for each formula $\delta
\in \Delta$, there exists a formula $\gamma \in \Gamma$ such that $\pi(\delta)
= \gamma$.

If a formulae $\Gamma$ is an instance of the numerator of a rule $R$, we say
that the rule $R$ can be applied to the formula $\Gamma$. Further, given a
partition $\pi$ which makes $\Gamma$ an instance of the numerator of a rule
$R$, we call
%
$$\Gamma \setminus \pi(\textrm{numerator}(R)) \cup
\pi(\textrm{denominator}(R))$$
%
an instance of the rule $R$ on a set of formulae $\Gamma$.
\end{definition}
\begin{definition}{Applicable Partition}\label{Applicable Partition}

An applicable partition of a set of formulae $\Gamma$ with respect to another
set of formulae $\Delta$ is a partition $\pi$ where $\Delta$ is an instance of
the set with respect to the partition $\pi$.
\end{definition}
\begin{definition}{Invertible Rule}\label{Invertible Rule}

A rule $\rho$ is invertible if and only if whenever there exists a closed
tableau for an instance of its numerator, there exists a closed tableau for
each branch in its denominator.
\end{definition}
%
\subsection{Classical Propositional Logic}
%
We define classical propositional logic (CPL) as the following.
%
\begin{definition}{Syntax of CPL}\label{Syntax of CPL}

Given a finite set of propositional symbols, $PV$, a formula in CPL is
described by the following grammar (where $p \in PV$).
%
\begin{equation*}
\varphi \; ::= \; p \; \vert \; \top \; \vert \; \bot \; \vert \; \neg
(\varphi) \; \vert \; (\varphi \wedge \varphi) \; \vert \; (\varphi \vee
\varphi) \; \vert \; (\varphi \rightarrow \varphi)
\end{equation*}
%
We also define the other standard connectives for CPL as the following.
%
$$
\begin{array}{rlcll}
(\varphi \leftrightarrow \psi) &&\equiv&& ((\varphi \rightarrow \psi) \wedge
(\psi \rightarrow \varphi)) \\
\end{array}
$$
%
\end{definition}
\begin{definition}{CPL Models and Valuations of CPL Formulae}\label{CPL Model}

We define a CPL Model to be a function $\vartheta \, : \, PV \rightarrow
\{True, False\}$ \cite{kelly2009revised}. The valuation of a formula under a
model $\vartheta$ is defined recursively as the following.
%
\begin{align*}
\vartheta (\bot) &= False \\
\vartheta (\top) &= True \\
\vartheta (\neg(\varphi)) &=
\begin{cases}
True & \textrm{If } \vartheta(\varphi) = False \\
False & \textrm{otherwise} \\
\end{cases} \\
\vartheta (\varphi \wedge \psi) &=
\begin{cases}
True & \textrm{If } \vartheta(\varphi) = True \textrm{ and } \vartheta(\psi) =
True \\
False & \textrm{otherwise} \\
\end{cases} \\
\vartheta (\varphi \vee \psi) &=
\begin{cases}
True & \textrm{If } \vartheta(\varphi) = True \textrm{ or } \vartheta(\psi) =
True \\
False & \textrm{otherwise} \\
\end{cases} \\
\vartheta (\varphi \rightarrow \psi) &=
\begin{cases}
True & \textrm{If } \vartheta(\varphi) = False \textrm{ or } \vartheta(\psi) =
True \\
False & \textrm{otherwise} \\
\end{cases} \\
\end{align*}
%
\end{definition}
\begin{definition}{Satisfiability in CPL}\label{Satisfiability in CPL}

A CPL formula $\varphi$ is satisfiable if and only there exists a CPL model
$\vartheta$ such that $\vartheta(\varphi) = True$.
\end{definition}
\begin{definition}{Validity in CPL}\label{Validity in CPL}

A CPL formula $\varphi$ is valid if and only for all CPL models
$\vartheta$, $\vartheta(\varphi) = True$.

It follows that in CPL, a formula $\varphi$ is valid if and only if its
negation $\neg \varphi$ is not satisfiable.
\end{definition}
%
We will omit parentheses for CPL formulae for convenience and clarity. The
precedence will be from highest to lowest as defined in the definition.
%
\section{Tableau for CPL}\label{Tableau for CPL}
%
The tableau calculus we use for CPL requires formulae to be in negation
normal form.
%
\begin{definition}{Negation Normal Form for CPL}\label{NNF for CPL}
Negation normal form for CPL, is a form for formulae such that they only
consist of connectives in the following set, $\{\bot, \, \neg, \, \vee,\,
\wedge\}$. We define the negative normal form of a formulae inductively.
%
$$
\begin{array}{rlcll}
\nnf(\top)&\quad&=&\quad&\top \\
\nnf(\neg\top)&&=&&\bot \\
\nnf(\bot)&\quad&=&\quad&\bot \\
\nnf(\neg\bot)&&=&&\top \\
\nnf(p)&&=&&p \\
\nnf(\neg p)&&=&&\neg p \\
\nnf(\neg\neg\varphi)&&=&&\nnf(\varphi) \\
\nnf(\varphi \wedge \psi)&&=&&\nnf(\varphi) \wedge \nnf(\psi) \\
\nnf(\neg(\varphi \wedge \psi))&&=&&\nnf(\neg\varphi) \vee \nnf(\neg\psi) \\
\nnf(\varphi \vee \psi)&&=&&\nnf(\varphi) \vee \nnf(\psi) \\
\nnf(\neg(\varphi \vee \psi))&&=&&\nnf(\neg\varphi) \wedge \nnf(\neg\psi) \\
\nnf(\varphi \rightarrow \psi)&&=&&\nnf(\neg\varphi \vee \psi) \\
\nnf(\neg(\varphi \rightarrow \psi))&&=&&\nnf(\varphi) \wedge \nnf(\neg\psi)
\end{array}
$$
%
It is known that for every CPL formulae, there is an equivalent negation normal
form formula.
\end{definition}
\begin{definition}{Tableau Calculus for CPL}\label{Tableau Calculus for CPL}

Let the following rule set define the tableau calculus for CPL which we will be
using in this paper.
%
\begin{multicols}{4}
\noindent
\begin{equation*}
(\bot)\frac{\bot\,;\,Z}{\times}
\end{equation*}
\begin{equation*}
(Id)\frac{p\,;\,\neg p\,;\,Z}{\times}
\end{equation*}
\begin{equation*}
(\wedge)\frac{\varphi \wedge \psi\,;\,Z}{\varphi\,;\,\psi\,;\,Z}
\end{equation*}
\begin{equation*}
(\vee)\frac{\varphi \vee \psi\,;\,Z}{\varphi\,;\,Z\,|\,\psi\,;\,Z}
\end{equation*}
\end{multicols}
%
It can be shown that these rules are invertible.
\end{definition}
%
\section{Implementation in Coq}
%
In this section we detail the main restrictions we have in defining the
framework in Coq instead of a traditional programming language. The framework
is similar to the Tableau Work Bench which is implemented in O'Caml
\cite{abate2007tableau, kelly2009revised}.
%
\subsection{Defining Functions} \label{Defining Functions}
%
Coq defines a dependently typed functional programming language. However,
unlike programming languages not in an interactive theorem prover (like O'Caml
and Haskell), Coq requires the decidability of type-checking for all defined
functions to allow for proofs \cite{barthe2006defining}. As a result, all
functions defined in Coq are required to be provably terminating, total and
deterministic. Totality and deterministic functions are enforced through the
Coq language when defining functions. However, to enforce termination, Coq
checks the definition of the function to ensure that recursive calls of the
function have a structurally (strictly) decreasing argument. For simple
functions, sometimes the termination of a function can be automatically
inferred by Coq. However, in general this requires the user to explicitly
provide a proof of termination.

To prove termination of general recursive functions, the functions are defined
using well-founded recursion where a proof of termination needs to be given.
Defining a function this way requires two steps. The first is to prove that a
relation is well-founded. The second is to prove that the function has a
decreasing argument, with respect to the well-founded relation in the first
part, in each of its recursive calls.

To prove a relation $R$ is well-founded in Coq, we prove in Coq that
\\\verb+well_founded R+ is a theorem. Underlying this in Coq is the notion of
the accessibility of a relation, defined as \verb+Acc+ in Coq.  The definition
of a well-founded relation in Coq is equivalent to the definition that the
relation does not have any infinite chains. This follows the intuition that
there are no infinitely nested recursive calls in the function, that is the
function terminates. Once the relation is proven to be well-founded, a general
recursive function can be defined with the \verb+Fix+ keyword in Coq, where a
proof that the recursive calls has a decreasing argument with respect to the
well-founded relation is required.

In the implementation of the tableau-based theorem prover synthesiser, we use
the well-founded relation for the depth of a tree to define various functions.
We define the relation as the following.
%
\comment{
we use
two main well-founded relations to define general recursive functions: the
length order of a list and the depth order of a tree like data structure.
We define the relations as the following.

\begin{definition}{Length Order Relation (LOR)}

We define the length order relation with respect to the length of lists.
That is for lists $l_1$ and $l_2$,

\begin{equation}
l_1 \preceq_{\textrm{LOR}} l_2 \iff \textrm{the
length of } l_1 \le \textrm{the length of } l_2
\label{lengthOrder}
\end{equation}
\end{definition}
}
%
\begin{definition}{Depth Order Relation (DOR)}

We define the depth order relation with respect to the depth of trees where the
depth of a tree is defined as the maximum number of edges between the root node
and the tree's leaves.
That is for trees $t_1$ and $t_2$,
%
\begin{equation}
t_1 \preceq_{\textrm{DOR}} t_2 \iff \textrm{the
depth of } t_1 \leq \textrm{the depth of } t_2
\label{depthOrder}
\end{equation}
\end{definition}
%
We prove that these relations are well-founded with respect to the specific
data structures we define in Coq.
%
\subsection{Code Extraction}
%
Coq's ability to extract programs from proofs and functions in a theorem prover
is one of the major motivations for implementing the synthesiser for
tableau-based theorem provers in Coq. The advantages of extracting programs
using Coq's extraction mechanism is that any property proven in Coq will still
hold true after extraction. With this property, we can generate certified
tableau-based theorem provers using the framework where the theorem provers
generated are guaranteed to express the logic the input rules describe, once
a proof is given \cite{letouzey2008extraction}. An important consideration for
the implementation of the framework is to define the data structures in
\verb+Type+ and not \verb+Prop+ as proofs and functions defined in \verb+Prop+
will be removed after extraction.
%
\section{Data Structures}
%
In implementing the tableau-based theorem prover synthesiser, we define general
notions of tableau and CPL in Coq. Additionally, we define a specific
implementation of a tableau-based theorem prover for CPL with the rule set
defined in section (\ref{Tableau for CPL}).
%
\subsection{CPL in Coq}\label{CPL in Coq}
%
We define CPL in Coq similarly to definition (\ref{Syntax of CPL}) based on
\cite{van2015propositional}.
%
\begin{verbatim}
Inductive PropF : Type :=
  | Var  : string -> PropF
  | Bot  : PropF
  | Conj : PropF -> PropF -> PropF
  | Disj : PropF -> PropF -> PropF
  | Impl : PropF -> PropF -> PropF.
\end{verbatim}
%
We further provide notation for the negation connectives for propositional
logic with respect to the definition of \verb+PropF+.
%
\begin{verbatim}
Definition Neg A := Imp A Bot.
\end{verbatim}
%
We define a propositional variable as a map from a \verb+string+ to a
\verb+PropF+. We note that we need to define a propositional variable as a map
from a type that is decidable with respect to  equality to \verb+PropF+
formulae. We use the pre-defined lemma in Coq, \verb+string_eq_dec+ to prove
that \verb+PropF+ is decidable with respect to equality. \verb+PropF+ needs to
be decidable with respect to equality as we need to compute comparisons of
\verb+PropF+ formulae in the framework defined.
%
\subsection{Tableaux in Coq}
%
Using the defined notion of CPL in section (\ref{CPL in Coq}), we define
general data structures to express a tableau calculus. We define the rules of a
tableau calculus and nodes in tableau as the following.
%
\begin{verbatim}
Inductive Denom :=
  | Terminate.
Definition PropFSet := list PropF.
Definition Numerator := PropFSet.
Definition Denominator := sum (list PropFSet) Denom.

Definition Rule := prod Numerator Denominator.
Definition TableauNode := sum PropFSet Denom.
\end{verbatim}
%
Where \verb+prod+ defines a product type and \verb+sum+ defines a sum type.

Notably, we define \verb+Rule+ in this context to be the respresentation of a
numerator and denominator of a tableau rule which only contains the principal
formulae. This becomes important when we try and generate the applicable of a
rule with respect to a \verb+PropFSet+.

We further explicitly define the derivation tree of a proof in the tableau
calculus similarly to \cite{dawson2003new}.
%
\begin{verbatim}
Inductive DerTree :=
  | Clf : DerTree
  | Unf : PropFSet -> DerTree
  | Der : PropFSet -> Rule -> list DerTree -> DerTree.
\end{verbatim}
%
This data structure is similar to a general rose-tree. The main difference is
that we distinguish between two types of leaves. \verb+Clf+ represents a closed
branch in the tableau tree. As we are attempting to show the unsatisfiability
of the formula in the root node, reaching this type of leaf signifies that no
additional applications of rules need to be done. \verb+Unf+ represents a node
in a tableau tree in which no rule has been applied to it. \verb+Der+
represents an inner node of a tableau tree. This type of node holds information
on the set of formula it had, the rule that was applied to it, and the children
it generates from that set of formula and rule.

It should be noted that \verb+Unf+ does not mean that the node in the tableau
cannot be made open. A tableau with a \verb+Unf+ node is defined as opened if
there exists no rule in the tableau calculus which can expand that \verb+Unf+
node.  Similarly we define a tableau to be closed if there exists a
\verb+DerTree+ that is generated through correct rule application and all of
its leaves are \verb+Clf+.

However, as \verb+DerTree+ was defined recursively as a list of
\verb+DerTree+s, Coq was unable to automatically define a useful inductive
scheme. We use the following induction scheme instead when using induction on
the \verb+DerTree+ type.
%
\begin{verbatim}
Fixpoint DerTree_induction
  (PT : DerTree -> Type)
  (PL : list DerTree -> Type)
  (f_Clf : PT Clf)
  (f_Unf : forall x, PT (Unf x))
  (f_Der : forall x r l, PL l -> PT (Der x r l))
  (g_nil : PL nil)
  (g_cons : forall x, PT x -> forall xs, PL xs -> PL (cons x xs))
  (t : DerTree) : PT t.
\end{verbatim}
%
This general induction scheme requires two predicates: one which works on a
\verb+DerTree+ and one which works on a list of \verb+DerTree+s.
Using this induction scheme, we are able to use the standard sub-tree induction
and depth induction when completing proofs in Coq.
%
\comment{
Using this,
we use the general induction scheme to define induction on the \verb+DerTree+
type as the standard structural induction on a generic rose-tree. That is 
induction where we prove the proposition holds for leaves as the base case, and
for the inductive step we assume that the proposition holds for all elements
of a list of \verb+DerTree+s and prove that it holds for a node with that list
of \verb+DerTree+s as its children.
We also use this induction scheme to define induction on the depth of a
rose-tree. For the base case we prove that the proposition holds for the case
where the \verb+DerTree+ is a leaf node again. For the inductive case we prove
that the proposition holds for a node when assuming that it hold for all
\verb+DerTree+s with depth less than it.
}

We define the depth of a \verb+DerTree+ as one would expect.
%
\begin{verbatim}
Fixpoint depthDerTree (T : DerTree) :=
  match T with
  | Clf              => 0
  | Unf _            => 0
  | Der _ _ branches => 1 + maxList (map depthDerTree branches)
  end.
\end{verbatim}
%
Where \verb+maxList+ takes the maximum natural number in a list and \verb+map+
is defined as the usual map function in functional programming languages.
%
\section{Tableau-based Theorem Prover}\label{Tableau-based Theorem Prover}
%
To determine the satisfiability of a formula using the tableau method, tableau
rules must be applied to determine if the tableau generated by the formula can
be made closed.  At each application of a rules, three main considerations need
to be made \cite{abate2007tableau, kelly2009revised}. We denote any non-closed
leaf in a tableau tree to be a goal of a tableau tree.
%
\begin{itemize}
\item \textbf{Node-choice:} determining which goal of a tableau to apply a rule
to.
\item \textbf{Rule-choice:} determining which rule to apply a goal.
\item \textbf{Formula-choice:} determining which formula in a goal to apply
a rule to.
\end{itemize}
%
Similarly, we create functions in Coq to allow the user to make each of these
decisions. We note that any node in a \verb+DerTree+ which is of the type
\verb+Unf+ is a goal of the \verb+DerTree+. To define the functions to allow
the user to make these choices, we define how to generate applicable
partitions, how to apply a rule to a tableau node, and how to traverse a
\verb+DerTree+ to make the choices in Coq.
%
\subsection{Generating Applicable Partitions}
%
When we apply a rule to a node of a tableau tree, we wish to generate all
possible applications of the rule. It follows that each possible application of
the rule is a formula-choice. To define rule application in Coq, the notion of
a partition in definition (\ref{Partition}) is encoded as a list of pairs.
%
\begin{verbatim}
Definition Partition := list (PropF * PropF).
\end{verbatim}
%
Given a \verb+Partition+ $\pi$, the associated partition map is the map defined
by each element of $\pi$. If $(a, \, b)$ is an element of $\pi$, then in the
map of $\pi$, $a$ maps to $b$. Given the partial function defined in this way,
if a \verb+PropF+ formula is not in the domain of the function, it is mapped to
itself to give us a total function. For a \verb+Partition+ $\pi$ we denote the 
partial function generated by $\pi$ as $\mathcal{F}^{P}_{\pi}$ and the total
function generated by $\pi$ as $\mathcal{F}^{T}_{\pi}$.

We additionally define what it means to create a well-formed partition.
%
\begin{enumerate}
\item An empty partition is well-formed.
\item Let \verb+xs+ be a well-formed partition. \verb+(a, b) :: xs+ is
well-formed if \verb+a+ is not in the domain or the co-domain of
$\mathcal{F}^{P}_{\verb+xs+}$.
\end{enumerate}
%
Furthermore, we define a well-formed extension of a well-formed partition
$\pi_1$ with another well-formed partition $\pi_2$ as the following. The
extension of the partition $\pi_1$ with respect to the partition $\pi_2$ is
the list $\pi_1$ appended to the list $\pi_2$. This extension is well-formed
if $\textrm{Image}(\mathcal{F}^{P}_{\pi_1}) \cap
\textrm{Domain}(\mathcal{F}^{P}_{\pi_2}) = \emptyset$ and
$\textrm{Domain}(\mathcal{F}^{P}_{\pi_1}) \cap
\textrm{Domain}(\mathcal{F}^{P}_{\pi_2}) = \emptyset$. This is equivalent to
the notion of repeatedly adding elements to form a well-formed partition.

We define a function \verb+extendPartition+ in Coq to define well-formed
partition extension.
%
\begin{verbatim}
Fixpoint extendPartition (p1 p2 : Partition) : option Partition.
\end{verbatim}
%
The function returns \verb+None+ if the extension results in a non-well-formed
partition, otherwise it returns the well-formed extension.

We define \verb+partition_help+ in Coq,
%
\begin{verbatim}
Fixpoint partition_help (scheme : PropF) (propset : PropFSet)
  (pi : Partition) : list (Partition).
\end{verbatim}
%
\verb+partition_help+ generates all applicable partition tuples of a formula
\verb+scheme+ with respect to a set of formulae \verb+propset+.
The function then returns all well-formed extensions of \verb+pi+ with respect
to each applicable partition tuple generated to form a list.
If all the applicable partitions are unable to extend \verb+pi+ to give a
well-formed partition, the function returns an empty list.

With \verb+partition_help+ we iteratively define a process to generate all
applicable partition tuples of a set of formulae with respect to another set of
formulae. \verb+getPartitions+ is used to find the applicable partitions of a
numerator of a rule with respect to a node of the tableau tree.
%
\begin{verbatim}
Fixpoint getPartitions_help (schema propset : PropFSet)
  (acc : Partition) : list Partition :=
    match schema with
    | nil   => acc :: nil
    | s::ss => flat_map
                 (fun pi => getPartitions_help ss propset pi) 
                   (partition_help s propset acc)
    end.

Definition getPartitions (schema propset : PropFSet) :=
  getParititons_help schema propset acc.
\end{verbatim}
%
Where \verb+flat_map+ is provided by Coq.
%
\subsection{Tableau Rule Application}
%
Given a \verb+Rule+ and a \verb+Partition+ we define a function which
creates a \verb+DerTree+ inner node with respect to a \verb+PropFSet+.

We first define \verb+applyPartition+, which evaluates the map defined by a
partition on a set of formulae.
%
\begin{verbatim}
Fixpoint applyPartition (propset : PropFSet) (pi : Partition).
\end{verbatim}
%
This is simply the element-wise evaluation of $\mathcal{F}_{\verb+pi+}^{T}$ on
a \verb+propset+. We similarly define \verb+denoApply+ which additionally takes
into account the case in which the denominator of a rule is a terminating
symbol. In this case, \verb+denoApply+ returns a terminating symbol.
\comment{
We also define the notion of set subtraction in Coq, \verb+removeMultSet+.
%
\begin{verbatim}
Fixpoint removeMultSet (remove setprop : PropFSet) : PropFSet.
\end{verbatim}
%
This function simply returns a \verb+PropFSet+ which is the set \verb+setprop+
subtracted by the set \verb+remove+.
}

Additionally we define the following function.
%
\begin{verbatim}
Fixpoint derTreeAppend (rule : Rule) (propset : PropFSet)
  (branches : list TableauNode) (acc : list DerTree)
  : option DerTree.
\end{verbatim}
%
\verb+derTreeAppend+ defines a inner node with \verb+rule+ and set
\verb+propset+ with the children generated by \verb+branches+ (with an
accumulator to turn \verb+TableauNode+s into \verb+DerTree+s).

With these functions, we define \verb+applyPartitionRuleD+ which makes a
\verb+DerTree+ inner node which is the result of a tableau rule being applied
to a \verb+PropFSet+ (where the formula-choice of the rule application is
defined by a given partition).
%
\comment{
\begin{verbatim}
Definition applyPartitionRuleD (rule : Rule) (propset : PropFSet)
  (pi : Partition) : option DerTree :=
    let inst := applyPartition (getNumerator rule) pi in
    let X := removeMultSet inst propset in
    match pi with
    | nil => None
    | _ => match (getDenominator rule) with
           | inr res => Some Clf
           | res     => derTreeAppend rule propset
                          (tableauAppend X (denoApply pi res)) nil
           end
    end.
\end{verbatim}
}
\begin{verbatim}
Definition applyPartitionRuleD (rule : Rule) (propset : PropFSet)
  (pi : Partition) : option DerTree.
\end{verbatim}
%
The function assumes that the argument \verb+pi+ is a partition which will
correctly allow the set of formulae \verb+propset+ be applied to tableau rule
\verb+rule+. The choice of partition \verb+pi+ determines the formula-choice
when the rule is applied to the propset. Assuming that the arguments allow for
valid rule application, if the denominator of \verb+rule+ is a terminating
symbol, \verb+applyPartitionRuleD+ returns a \verb+Clf+ node. Otherwise, the
function will return a \verb+Der+ node with \verb+propset+, \verb+rule+ and a
list of children \verb+Unf+ nodes. The children nodes are the result of
\verb+rule+ being applied to \verb+propset+ (with the formula-choice being
decided by \verb+pi+).
%
\subsection{Traversing DerTree}
%
To expand a tableau through rule application, we must traverse the
\verb+DerTree+ to make a node-choice. To do this we define the function
\verb+getGoals+ which returns all the goals of a \verb+DerTree+.
%
\begin{verbatim}
Fixpoint getGoals (T : DerTree) : list DerTree.
\end{verbatim}
%
We define a function which will identify which branch in a list of branches
(which is in order from left-to-right in a tableau) has the $n^{\textrm{th}}$
goal.
%
\begin{verbatim}
Definition traverseToNG (Ts : list DerTree) (n : nat) :
  option (list DerTree * DerTree * list DerTree * nat)
\end{verbatim}
%
The function returns a tuple containing the following: the branches to the left
of the branch which contains the $n^{\textrm{th}}$ goal, the branch which
contains the $n^{\textrm{th}}$ goal, the branches to the right of the branch
which contains the $n^{\textrm{th}}$ goal, and the updated number of the goal
with respect to the branch containing the goal.

We define a function which applies a function of type
\verb+DerTree -> DerTree+ to the $n^{\textrm{th}}$ goal of a \verb+DerTree+.
%
\begin{verbatim}
Definition toBranchNG (T : DerTree) :
  (DerTree -> DerTree) -> nat -> DerTree.
\end{verbatim}
%
We also define a function to find the $n^{\textrm{th}}$ goal in a
\verb+DerTree+.
%
\begin{verbatim}
Definition getNGoal (T : DerTree) : nat -> option DerTree.
\end{verbatim}
%
Both \verb+toBranchNG+ and \verb+getNGoal+ was defined using well-founded
induction using the depth order relation (\ref{depthOrder}).

\comment{
Additionally we define a map function for option types.
%
\begin{verbatim}
Definition optionSucMap (A B : Type) (f : A -> option B)
  (xs : list A) : option (list B).
\end{verbatim}
%
\verb+optionSucMap+ applies \verb+f+ to each element in \verb+xs+. For each of
the results which do not return \verb+None+, the result is appended to a list
which is the return value of the function.  If the final list is empty,
\verb+None+ is returned.

We also define the following simple function which returns a function which
returns a \verb+DerTree+.
%
\begin{verbatim}
Definition updateLeaf (T : DerTree) :
  (DerTree -> DerTree) := fun _ => T.
\end{verbatim}
%
}
We define the function to apply a rule to the $n^{\textrm{th}}$ goal of a
\verb+DerTree+.
%
\begin{verbatim}
Definition applyRtoNG (T : DerTree) (rule : Rule) (n : nat).
\end{verbatim}
\comment{
\begin{verbatim}
Definition applyRtoNG (T : DerTree) (rule : Rule) (n : nat) :=
  match getNGoal T n with
  | None      => None
  | Some goal => 
      match goal with
      | Unf propset =>
        match getPartitions (getNumerator rule) propset with
                 | nil => None
                 | res => match optionSucMap
                   (applyPartitionRuleD rule propset) res with
                     | None          => None
                     | Some newNodes => Some
                       (map (fun x => toBranchNG T x n)
                       (map updateLeaf newNodes))
                     end
                 end
      | _ => None
      end
  end.
\end{verbatim}
}
%
\verb+applyRtoNG+ generates all possible ways a \verb+Rule+ can be applied to
the $n^{\textrm{th}}$ goal of a \verb+DerTree+. The function gets the
$n^{\textrm{th}}$ goal of the \verb+DerTree+ \verb+T+ using \verb+getNGoal+. It
then generates all possible applicable partitions of the \verb+Rule+
\verb+rule+ with respect to the set of formulae \verb+propset+ with
\verb+getPartitions+.  \verb+applyPartitionRuleD+ is then used to generate new
nodes which are the $n^{\textrm{th}}$ goal applied with the tableau rule
\verb+rule+ (with respect to the formula-choice chosen by each partition
generated by \verb+getPartitions+). The function then returns a list of the
\verb+DerTree+ T with the goal node replaced by each of the new nodes
generated.

The function allows the user to make a rule-choice and a node-choice when
applying a rule to the tableau. The remaining choice to be made is the
formula-choice when applying a rule. \verb+applyRtoNG+ returns a list of all
possible applications of a rule on the $n^{\textrm{th}}$ goal of a tableau.
Thus to account for a formula-choice, we just pick from the returning list of
\verb+applyRtoNG+. We define the following function to select from a list of
results.
%
\begin{verbatim}
Fixpoint pickNFApply (results : option (list DerTree)) (n : nat)
  : DerTree.
\end{verbatim}
%
\subsection{CPL Example}
%
We demonstrate how \verb+applyRtoNG+ and \verb+pickNFApply+ can be used to
prove the validity of formulae in CPL by proving that the following formula is
valid.
%
\begin{equation}
(A \wedge (A \implies B)) \implies B
\label{cpl example formula}
\end{equation}
%
We convert the tableau rules for CPL detailed in section (\ref{Tableau for
CPL}) to the \verb+Rule+ type in Coq, only including the principal formulae of
each tableau rule. We alias each of the rules as the following in Coq.
%
\begin{verbatim}
Inductive CRule := 
  | BotC : CRule
  | IdC  : CRule
  | OrC  : CRule
  | AndC : CRule.
\end{verbatim}
%
We define the function \verb+getCRule+ of type \verb+CRule -> Rule+. This
function returns the \verb+Rule+ tuple associated to each of the tableau rules
aliased by the inductive type \verb+CRule+. We define \verb+applyCRtoNG+ to be
a specific implementation of \verb+applyRtoNG+ for \verb+CRule+. To determine
the validity of (\ref{cpl example formula}), we look at the formula's negation.
The negation of (\ref{cpl example formula}) is converted into negation normal
form such that the tableau calculus defined can be used to check the
satisfiability of the formula. The procedure to convert the negation of
(\ref{cpl example formula}) is defined in definition (\ref{NNF for CPL}).
%
\begin{equation}
A \wedge ((\neg A \vee B) \wedge \neg B)
\label{cpl example nnf}
\end{equation}
%
Thus, with (\ref{cpl example nnf}), we have the following proof in Coq for
tableau.
%
\begin{verbatim}
Definition cpl_example :=
  ((# "A") ∧ ((¬ (# "A")) ∨ (# "B")) ∧ (¬ (# "B"))).

Definition step1 := Unf (cpl_example :: nil).
Definition step2 := pickNFApply (applyCRtoNG step1 AndC 1) 1.
Definition step3 := pickNFApply (applyCRtoNG step2 AndC 1) 1.
Definition step4 := pickNFApply (applyCRtoNG step3 OrC 1) 1.
Definition step5 := pickNFApply (applyCRtoNG step4 IdC 1) 1.
Definition step6 := pickNFApply (applyCRtoNG step5 IdC 1) 1.
\end{verbatim}
%
We can check the process of each step by using \verb+Compute+ with the function
\verb+getGoals+ to check what is needed to be proven at each step, that is the
leaves of the tableau tree which are not closed.
By running \verb+Compute getGoals step6.+ we get \verb+= nil : list DerTree+ as
the output showing that the tableau is closed. Thus it follows that as the
negation of (\ref{cpl example formula}) results in a closed tableau, it is 
unsatisfiable. Thus as expected (\ref{cpl example formula}) is shown to be a
theorem.
%
\section{Future Work}
%
In this paper, we have described a framework in Coq which can be extracted to
give functions which allows the user to guide a tableau proof. This system this
provides is an interactive theorem prover. There are two logical ways to extend
the framework: first is to add automation and second is to add support for
additional logics.

To extend the framework for automation, we need to consider the node-choice,
rule-choice and formula-choice as defined in section (\ref{Tableau-based
Theorem Prover}). We can follow methods used by the Tableau Work Bench
\cite{abate2007tableau} to make decisions about these choices.  Node-choice can
be handled by a depth-first search, in the tableau when a rule is applied, the
left most branch is searched first. Formula-choice is only important if the
rules are invertible. If a rule is not invertible (definition (\ref{Invertible
Rule})), a backtracking approach to rule application is needed. We further
discuss the use of backtracking below when dealing with additional logics. To
determine the rule-choice, the Tableau Work Bench requires users to input a
strategy. This directs the proof search of the Tableau Work Bench, guiding
which rule should be applied at a particular step in a tableau proof. That is,
in the Tableau Work Bench the user gives the system a strategy in which rules
will be evaluated to expand a tableau. An additional requirement of
implementing automation in this framework would be to ensure that the proof
search method terminates. Thus for the implementation in Coq, for each strategy
a user provides, the proof search process of the tableau specified by the
strategy must be proven to terminate by either the framework or the user. In
general, proving that a strategy is terminating automatically cannot be
automated. The process requires defining a well-founded relation in which the
proof search function decrements at each recursive call, section (\ref{Defining
Functions}).

To extend the framework for logics other than CPL, we must deal with rules
which are not invertible. As mentioned above, formula-choice needs to be
considered when rules are not invertible. If the application of an invertible
rule leads to a closed tableau, the choice of formula in which the rule is
applied to does not matter. However, in the case in which a tableau rule is not
invertible, an incorrect selection of formulae in the application of a rule can
cause a branch of the tableau to be impossible to close, even if there exists a
combination of rule application which can make the branch closed. Thus, to
ensure that a branch is truly unable to be closed, all combinations of
formula-choice for a non-invertible rule must be tested. This can be done
through backtracking. Whenever, a non-invertible rule results in an unclosed
branch where no additional rules can be applied, the tableau backtracks. The
tableau goes back to the state of the non-invertible rule and a different
formula-choice is made until it either gives a closed branch, or there are no
more formula-choices to be made, confirming that the branch is indeed opened.
%
\section{Conclusion}
%
In this paper we describe the first steps in implementing a framework which
allows users to synthesise tableau-based theorem provers for various logics.
We outline the differences which need to be considered when implementing the
framework in Coq instead of typical programming languages.  Specifically, we
demonstrate how the framework can be used to make a tableau-based theorem
prover for classical propositional logic and how it can be used to prove that a
formula is a theorem in classical propositional logic.
%
\bibliography{reference}
\bibliographystyle{splncs}
%
\end{document}
