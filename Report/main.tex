\documentclass{llncs}

\usepackage[utf8]{inputenc}
\usepackage[margin=1in]{geometry}

\begin{document}

\title{Automated Synthesis of a Tableaux Theorem Prover for Classical
Propositional Logic using Coq}

\author{Alexander Soen \and Rajeev Goré}

\maketitle

\begin{abstract}
Proving or disproving that a formula is a theorem in a logical system is an
essential process in many areas of applied logic. Theorem provers which utilise
the tableau method is a common method in which this is done for a variety of
logics. We describe a framework in Coq which can be used to synthesise
tableau-based theorem provers for various logics. The main novelty of using
this framework to implement a theorem prover is the following. Firstly, to
synthesis an interactive theorem prover using the framework, only the grammar
of the formula and rule set of the logics is needed. Secondly, that the
underlying implementation of the rule set used to define the logics of the
theorem prover is in Coq. Subsequently, this allows the framework and the user
to prove various properties about the synthesised theorem prover in Coq, such
as the correct correspondence between the rule set and the synthesis theorem
prover and the completeness and soundness of the rule set with respect to the
logical system being defined. We demonstrate this framework in Coq by
implementing the standard tableau calculus for classical propositional logic.
\end{abstract}

\section{Introduction}

Many different implementations of the tableau method exists within the
literature for specific logics. There exists some very effective domain
specific theorem provers such as Fact++ and MSPASS (REFERENCE). However,
whenever a new tableau calculus is devised, trying to develop a corresponding
theorem prover can be difficult without specific knowledge in programming
theorem provers. The current systems which try and 

\section{Preliminaries}

\end{document}
